%% Justin Bailey 2011
%% jgbailey@codeslower.com
%%
%% To use: surround paragraphs to place a rule
%% with \startrule and \endrule.
%%
%% E.g.:
%%
%% \startrule 
%% This paragraph will have a rule around it.
%% \endrule
%%
%% Multiple paragraphs can be spanned as well. Rules will break across
%% pages but it will have the effect of making your document
%% \raggedbottom. 
%%
%% You can change the offset of the rule by setting \ruleoffset. Rules
%% are always offset from the left margin.
%%
%% A simple \codeblock environment is included too.
%%
%% To use it, surround the code with \codeblock{
%%
%% }
%%
%% Code must appear in the group. The group must immediately follow \codeblock.
%% Code is set ragged right, obeying newlines, using typewriter font.

\newdimen\ruleoffset \ruleoffset=-2pt %% Horizontal offset for the
                                  %% rule. This is the parameter that
                                  %% should be set by the user.

\newbox\parbox %% Holds our collected paragraph
\newdimen\startpar \startpar=-1000pt %% Where the paragraph started
\newdimen\endpar  \endpar=0pt %% Where the paragraph ended
\newdimen\roffset %% For calculating offset from left margin for rule.
\newtoks\savedout \savedout=\output %% Previous output routine
\newdimen\interparskip

%% A hbox with no width, height or depth. Used to ensure
%% empty lines have at least one item, which prevents
%% underfull warnings.
\newbox\zerobox \setbox\zerobox=\hbox to 0pt{\vrule height0pt width0pt}
\newbox\charbox \setbox\charbox=\hbox{.}

%% Capture all paragraphs in a \vbox so we can keep 
%% track of their height. If this macro is not in vertical
%% mode, it will emit a \par to force vertical mode.
\def\startrule{\ifvmode\else\par\penalty0\fi%%
%% Capture value of \ruleoffset so it survives the group. This ensures
%% that when the output routine calls \makerule we get the right value
  \xdef\setroffset{\roffset=\the\leftskip \advance\roffset by \the\ruleoffset}%%
  \xdef\theprevdepth{\the\prevdepth}%%
  \startpar=\pagetotal \endpar=-1000pt%% Sentinel value.
  \setbox\parbox=\vbox\bgroup}

%% Insert a rule from the top of the paragraph to the end. If
%% the paragraph started on the previous page this will put a
%% rule only on the bottom portion.
\def\endrule{\egroup
  %% Do not calculated inter-paragraph glue if
  %% nothing exists on the page yet.
  \if\pagegoal=\maxdimen\relax\else%%
%% Calculate inter-paragraph glue by setting captured text
%% ourselves. We know the heights of everythign but the glue. Use
%% \vtop in boht cases so height of box is only the first line, not
%% the entire box.
    \setbox2=\vtop{X\par}%%
    \setbox4=\vtop{\unvcopy\parbox}%% 
%% Set \prevdepth so inter-paragraph glue is calculated based
%% on the paragraph that really preceded \startrule, not our
%% fake paragraph.
    \setbox0=\vbox{\copy2\par\prevdepth=\theprevdepth\copy4}%%
    \interparskip=\ht0 \advance\interparskip by -\ht4 \advance\interparskip by -\ht2%%
%% Add \penalty so \output routine triggers if necessary.
    \vskip\interparskip\par\penalty0%% 
  \fi%%
  \ifnum\startpar>-1000\relax%% A page break occurred when startpar = -1000.
%% \pagetotal represents where our paragraph will start. 
    \startpar=\pagetotal%%
  \fi%%
%% Add \penalty below so \output will run if necessary.
  \unvbox\parbox\par\penalty0%%
  \xdef\theprevdepth{\the\prevdepth}%%
  \endpar=\pagetotal%% Record where we are. 
%% Make the rule start at the top of the first line, extending to
%% the bottom of the last line.
  \if\endpar>\pagegoal%%
    \dimen0=\pagegoal%%
  \else%%
    \dimen0=\endpar%% 
  \fi%%
  \ifnum\startpar>-1000\relax%% when true, rule starts at top of page.
%% Otherwise, \startpar in middle of page, \endpar also in middle
    \advance\startpar by -\pagedepth%%
    \advance\dimen0 by -\startpar%%
  \fi%%
  \makerule\dimen0/%%    
  \endpar=0pt%% Prevents \lastrule from drawing a rule
  \startpar=-1000pt%%
  \prevdepth=\theprevdepth}

%% An output routine that handles breaking rules across
%% page boundaries. After adding a rule, runs the default (or prior)
%% output routine. 
\def\lastrule{%%
  \ifnum\endpar<0\relax%
%% Any glue in box255 will screw up \startpar - the paragraph might
%% have moved since we recorded its position. Therefore, we reset the
%% box to its natural height and put all the stretching at the
%% bottom. 
    \setbox0=\vbox{\unvcopy255} \dimen0=\ht0%%
    \ifnum\startpar>-1000\relax%%
    %% \startpar in middle of page, \endpar at bottom.
      \advance\startpar by -\dp0%% Make sure rule covers the entire paragraph.
      \advance\dimen0 by -\startpar%%
      \wlog{output: dimen0: \the\dimen0}%%
    \fi%%
    \setbox255=\vbox{\unvbox255\makerule\dimen0/\vfil}%%
    \global\startpar=-1000pt%%
  \fi%%
%% Run previous output routine.
  \the\savedout%%
}

%% Handles drawing the rule, using the height given in the dimension
%% register passed. Notice the slash used to delimit the
%% argument. Used internally by macros above.
\def\makerule#1/{%%
%% Keep prevdepth so interline spacing isn't affected by our vbox
%% (from
%% http://tex.stackexchange.com/questions/22355/make-an-invisible-vbox)
  \nointerlineskip%% No glue between previous vbox and this
  \nobreak\vbox to 0pt{\hrule height 0pt depth0pt width0pt\vskip -#1%% \hrule ensures reference point is at the top of the box.
%% This box contains the rule of the given height, offset from the
%% left margin by \ruleoffset.
        \hbox to 0pt{\setroffset%%
%% \hss removes an underfull warning
          \kern\roffset\vrule height#1 width1pt depth0pt\hss}}%%
%% Restoring prevdepth ensures next paragraph
%% is spaced correctly from previous one.
  \prevdepth=\theprevdepth}

%% Ensures spaces at the beginning of the line are always
%% preserved. TABs will not be. Thanks to TeX for the Impatient
%% (eplain) for \alwayspace.
{\gdef\alwaysspace{\hglue\fontdimen2\the\font \relax}%%
  \obeyspaces\gdef {\alwaysspace}}

%% Define new lines so that in \codeblock they don't start a new
%% paragraph - they just insert a line break.
{\catcode`\^^M=\active \gdef^^M{\copy\zerobox\hfil\break} \global\let\ret=^^M}

\newtoks\codetoks
%% \codeblock must be followed by a group or it has no effect.
%% When followed by a group, the text found will be set on
%% individual lines as they appear in the group (i.e. new lines 
%% are obeyed). The entire group will be have a rule next to it.
%% The group is also set in typewriter font, with ragged-right
%% margins.
%%
%% Note that text in the group is NOT set verbatim.
\def\codeblock{\codetoks={}%%
%% Removes final lineskip if one was there.
  \gdef\endo{\unpenalty\endrule\prevdepth=0pt\relax}%%
%% Removes initial newline, if one was there. Otherwise, reinsert the
%% token captured.
  \gdef\ignorenewline{\ifx\next\ret%%
    \else\next%%                      
    \fi}%%                          
  \gdef\do{\ifx\next\bgroup%% 
    \codetoks={\startrule\noindent\bgroup%%
      \ttraggedright%%
      \parindent=0pt%%
      \tt%%
      \aftergroup\endo%%
      \ignorespaces%%
      \catcode`\^^M=\active\obeyspaces%%
      \afterassignment\ignorenewline\let\next= }%%
    \fi\the\codetoks}%%
  \ignorespaces\afterassignment\do\let\next= }

\output={\lastrule} %% Use our custom output routine.

